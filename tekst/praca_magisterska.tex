\documentclass[12p]{report}
\usepackage[T1]{fontenc}
\usepackage[utf8]{inputenc}

\begin{document}

\tableofcontents
\newpage
\quad Celem pracy jest przedstawienie algorytmów kryptograficznych oraz możliwych ataków. Następnie implementacja 2 algorytmów na systemie wbudowanym oraz porównanie ich wydajności w zależności od rozmiaru szyfrowanych danych.
\newpage


%-----------------------------------
%	Rozdział 1
%-----------------------------------
\chapter{Kryptologia}
\section{Wprowadzenie}

\quad W obecnych czasach dużym zainteresowaniem cieszy się bezpieczeństwo cybernetyczne, którego szczególną częścią jest kryptografia. Szczególnie ważne zastosowanie znajduje w branży informatycznych, militarnej, urzędach, grupach developerskich czy bankowości. Kryptografia pojawiła się znaczniej wcześniej niż platformy obliczeniowe, zainteresowali się nią już ludzie z czasów starożytnych, pojawiła się wraz z umiejętnością pisania. Powodem istnienia kryptografii jest bezpieczne i prywatne dostarczanie wiadomości. Znajduje szczególne zastosowanie w przypadku danych przesyłanych drogą komunikacyjną. W obecnych czasach powszechną drogą komunikacyjną jest droga internetowa, dzięki kryptografii możliwe jest zapewnienie bezpieczeństwa cybernetycznego przesyłanych danych. W zależności od stopnia poufności informacji, którą chcemy zaszyfrować, aby niepożądane osoby jej nie odczytały można zastosować odmiennych algorytmów szyfrowania. 

\quad Kryptologia to połączenie kryptografii i kryptoanalizy. W języku greckim 'kryptos' oznacza ukryty, zaś 'logos' tłumaczone jest jako słowo. Kryptologia jest dziedziną zajmującą się ukrywaniem tekstu jawnego. Kryptografia jest dziedziną węższą od kryptologii, jest badaniem technik matematycznych związanych z bezpieczeństwem informacji. Do bezpieczeństwa danych można zaliczyć poufność informacji, uwierzytelnienie użytkowników i pochodzenia danych, a także integralność danych. Słowo kryptologia składa się z dwóch greckich słów: 'kryptos' znaczący ukryty i 'graph' oznaczający pisanie, jest to nauka o zabezpieczaniu danych. Za pomocą technik kryptograficznych możliwe jest zaszyfrowanie jawnego tekstu, w taki sposób aby niepożądana osoba nie mogła ich odczytać. Drugą gałęzią kryptologii jest kryptoanaliza, która zajmuje się analizą i możliwymi sposobami odszyfrowania kodu kryptograficznego.
\subsection{Terminologia kryptologiczna}
frfefe

\section{Maszyna Enigma}

\section{Szyfry blokowe i strumieniowe}
\section{Szyfry symetryczne i asymetryczne}
\subsection{Szyfry symetryczne}
\subsubsection{DES}
\subsubsection{AES}
\subsubsection{IDEA}
\subsubsection{Blowfish}
\subsection{Szyfry asymetryczne}
\subsubsection{RSA}
\subsubsection{DSS(DSA)}
\subsubsection{Diffie-Hellman}



\end{document}
